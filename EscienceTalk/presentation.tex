\documentclass{beamer}
\usepackage[danish,english]{babel}
\usepackage[latin1]{inputenc}
\usepackage{beamerthemesplit}
\usepackage{graphics,epsfig, subfigure}
\usepackage{url}
\usepackage{physics}
\usepackage{lmodern,textcomp}

\addtobeamertemplate{footline}
{%
	\usebeamercolor[fg]{author in sidebar}
	\vskip-1cm\hskip10pt
	%\insertpagenumber\,/\,\insertpresentationendpage\kern1em\vskip2pt%
	\insertframenumber\,/\,\inserttotalframenumber\kern1em\vskip2pt%
}



\definecolor{kugreen}{RGB}{50,93,61}
\definecolor{kugreenlys}{RGB}{132,158,139}
\definecolor{kugreenlyslys}{RGB}{173,190,177}
\definecolor{kugreenlyslyslys}{RGB}{214,223,216}
\setbeamercovered{transparent}
\mode<presentation>
{  \usetheme{PaloAlto}
  \usecolortheme[named=kugreen]{structure}
  \useinnertheme{circles}
  \usefonttheme[onlymath]{serif}
  \setbeamercovered{transparent}
  \setbeamertemplate{blocks}[rounded][shadow=true]
  
}

\makeatletter
\setbeamertemplate{sidebar \beamer@sidebarside}%{sidebar theme}
{
	\beamer@tempdim=\beamer@sidebarwidth%
	\advance\beamer@tempdim by -6pt%
	\insertverticalnavigation{\beamer@sidebarwidth}%
	\vfill
	\ifx\beamer@sidebarside\beamer@lefttext%
	\else%
	\usebeamercolor{normal text}%
	\llap{\usebeamertemplate***{navigation symbols}\hskip0.1cm}%
	\vskip2pt%
	\fi%
}%
\makeatother
%\setbeamertemplate{background}{\includegraphics[width=1\textwidth]{natfak_baggrund.pdf}}
\logo{\includegraphics[width=1.5cm]{SCIENCE_CMYK.pdf}}

\title{Implementing RISC-V\\ in Synchronous Message Exchange}
\author{Daniel Ramyar}
\institute{Niels Bohr Institute \\ University of Copenhagen}
\date{eScience lunch talk \\ October 2, 2019}

\begin{document}
	\begin{frame}
		\titlepage
		\vspace{-0.5cm}	
	\end{frame}

	\begin{frame}
		\frametitle{Overview}

		\small{\tableofcontents}%[pausesections]
	\end{frame}
	
	\section{Introduction}

	\begin{frame}
		\subsection{Motivation}
		\frametitle{Motivation}
        \begin{itemize}
            \item Most measurement instruments are based on Intel x86 CPU
            \item Limits the bandwidth at which data collection is possible
            \item Limits the possibility for custom solutions 
        \end{itemize}

	\end{frame}

    \begin{frame}
        \frametitle{Motivation}
        \begin{itemize}
            \item RISC-V especially well suited embedding in scientific instruments
            \item Fully customizable = faster and more power efficient  
            \item Could be implemented in a FPGA 
            \item But FPGA programming is complicated
        \end{itemize}
        
        \centering
        \vspace{1cm}
        \includegraphics[scale=0.35]{"Pictures and plots/tradeoffs_own"}
        
    \end{frame}

	\begin{frame}
        \subsection{Synchronous Message Exchange}
		\frametitle{Synchronous Message Exchange}
        
		\begin{itemize}
            \item Write FPGA applications within .Net framework\footnote{\tiny Building Hardware from C$\#$ models, Kenneth Skovhede and Brian Vinter}
            \item Enjoy productivity features of modern language
            \item Automatic VHDL conversion
            \item Based on Communicating Sequential Processes (CSP)\footnote{\vspace{0.5cm}\tiny Communicating sequential processes, Charles Antony Richard Hoare}
        \end{itemize}

	\end{frame}

    \begin{frame}
        \frametitle{Synchronous Message Exchange}
        
        \begin{itemize}
            \item Processes are isolated no memory sharing
            \item Communicates with each other through channels
            \item Globally synchronous
            \item Perfect for implementing RISC-V!
        \end{itemize}
    
        \begin{columns}[T] 
            \begin{column}[T]{5cm}
                \centering
                \begin{figure}
                    \includegraphics[scale=0.6]{"Pictures and plots/one_to_one"}
                    \caption{One to one}
                \end{figure} 
            \end{column}
            \begin{column}[T]{5cm} 
                \begin{figure}
                    \includegraphics[scale=0.5]{"Pictures and plots/one_to_many"}
                    \caption{One to many}
                \end{figure} 
            \end{column}
        \end{columns}
        
    \end{frame}

    \section{Implementing RISC-V in SME}

    \begin{frame}
        \subsection{Single Cycle RISC-V}
        \frametitle{Single Cycle RISC-V}
        \centering
        \vspace{-0.5cm}
        Full datapath for single cycle RISC-V
        \includegraphics[scale=0.35]{"Pictures and plots/FullDatapath"} \\
        \scriptsize{Taken from \textit{Computer organization and design RISC V}}
        
    \end{frame}
    
    \begin{frame}
        \subsection{Fetching instruction and increment}
        \frametitle{Fetching instruction and increment}
        \centering
        \vspace{-0.5cm}
        We need memory for instructions and a way to remember current instruction

        \includegraphics[scale=0.4]{"Pictures and plots/IFINC"} \\
        \scriptsize{Taken from \textit{Computer organization and design RISC V}}
        
    \end{frame}

    \begin{frame}
        \subsection{Instruction decode and execution}
        \frametitle{Instruction decode and execution}
        \centering
        \vspace{-0.5cm}
        Storage for data we currently work with and unit for execution for instructions
        
        \includegraphics[scale=0.4]{"Pictures and plots/RegALU"} \\
        \scriptsize{Taken from \textit{Computer organization and design RISC V}}
        
    \end{frame}

    \begin{frame}
        \frametitle{Instruction decode and execution}
        \centering
        \vspace{-0.5cm}
        Conventions according to RISC-V spec sheet
        
        \includegraphics[scale=0.5]{"Pictures and plots/registercon"} \\
        \scriptsize{Taken from \textit{The RISC-V Instruction Set Manual - Volume I: User-Level ISA}}
        
    \end{frame}

    \begin{frame}
        \subsection{Memory access}
        \frametitle{Memory access}
        \centering
        \vspace{-0.5cm}
        We need memory unit for storing data and immediate generation unit for calculating memory address
        
        \includegraphics[scale=0.4]{"Pictures and plots/MEMACC"} \\
        \scriptsize{Taken from \textit{Computer organization and design RISC V}}
        
    \end{frame}

    \begin{frame}
        \subsection{Branching}
        \frametitle{Branching}
        \centering
        \vspace{-0.5cm}
        Additional unit needed for branching instructions
        
        \includegraphics[scale=0.33]{"Pictures and plots/BRANCH"} \\
        \scriptsize{Taken from \textit{Computer organization and design RISC V}}
        
    \end{frame}

    \begin{frame}
        \subsection{Simple datapath}
        \frametitle{Simple datapath}
        \centering
        \vspace{-0.5cm}
        Wiring it up we have our simple datapath
        
        \includegraphics[scale=0.33]{"Pictures and plots/simpledatapath"} \\
        \scriptsize{Taken from \textit{Computer organization and design RISC V}}
        
    \end{frame}

    \begin{frame}
        \subsection{Instructions}
        \frametitle{Instructions}
        \centering
        \vspace{-0.5cm}
        \includegraphics[scale=0.28]{"Pictures and plots/instructions"} \\     
    \end{frame}

    \begin{frame}
        \frametitle{Simple datapath}
        \centering
        \vspace{-0.5cm}
        The R-Type instruction datapath would look like
        
        \includegraphics[scale=0.33]{"Pictures and plots/rformatpath"} \\
        \scriptsize{Taken from \textit{Computer organization and design RISC V}}
        
    \end{frame}

    \begin{frame}
        \subsection{Control}
        \frametitle{Control}
        \centering
        \vspace{-0.5cm}
        We need control unit to determine path for instructions.\\ 
        Main Control and ALU control are separated to simplify complexity of system. 
        
        \includegraphics[scale=0.3]{"Pictures and plots/control"} \\
        \scriptsize{Taken from \textit{Computer organization and design RISC V}}
        
    \end{frame}

    
    \section{Further work}
    
    \begin{frame}
        \frametitle{Further work}
        \begin{itemize}
            \item Implement rest of base instructions
            \item Pipeline the datapath
            \item Test on a FPGA
        \end{itemize}   
    \end{frame}
	

\end{document}