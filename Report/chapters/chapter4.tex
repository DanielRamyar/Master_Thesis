\chapter{Introduction to RISC-V instructions}
    To understand the design decisions made, building the RISC-V CPU in Chapter \ref{Chapter:5}, we will go through the fundamentals of the machine language behind the RISC-V architecture. 
    
    To this end we will go through a top down approach going from assembly code written by the programmer all the way down to the machine code representation in the hardware implementation. We will therefore start out by introducing the RISC-V assembly code in section \ref{section:RISC-V Asembly}. 
    
    Hereafter we will talk about the operands of the RISC-V CPU, which unlike high-level programming languages are confined in a small directory called the \textit{register}. 
    
    Then we will go through the base 2 numeral system to make meaning of the following section, which will go through how the assembly language, also known as instructions, is represented in the CPU itself. This chapter is based on chapter 2 in \cite{riscVbook}, which can be referred to if more detailed explanations are needed.

\section{RISC-V Assembly}\label{section:RISC-V Asembly}
    Like every computer, we need a way of telling it what to do. This is done via the RISC-V assembly language, which is a human readable abstraction of the RISC-V instruction set. There exists multiple instruction standards like \texttt{RV32I} and \texttt{RV64I} for the RISC-V architecture and extensions to these\footnote{The exect specifications can be found here \url{https://riscv.org/specifications/}}. This project will implement the \texttt{RV64I} instructions. For writing small RISC-V Assembler programs I can recommend the RISC-V Assembly simulator \textit{Jupiter}\footnote{\url{https://github.com/andrescv/Jupiter}}.
    
    For the assembler to know where to start reading instructions from we use:
    
    \begin{lstlisting}[language=]
.global __start

__start:
    \end{lstlisting}
    where all instructions are added after the \texttt{\_\_start:} command.
    
    We will in this example start by adding two numbers together. Keep in mind however that this is not executable by the Jupiter simulator, as crucial instructions have been omitted to ease readability for the reader. An executable version of the example will be shown when the necessary concepts have been introduced.
    
    To find the sum between two numbers, we use the \texttt{add} instruction:
    \begin{lstlisting}[language=, ]
.global __start

__start:
    add a, b, c
    \end{lstlisting}
    here we tell the CPU to find the sum between the two varibles, b and c, and save the result in variable a. It is worth noting that the RISC-V assembly language instructions always have exactly 3 operands and always performs a single operation.
    
    This may seem to restrictive, because how would the then calculate the following equation:
    \begin{equation}\label{Eq.Addexample}
        d = a + b + c
    \end{equation}
    
    The answer is to split the equation up in small bites. To add more instructions we simply add a new line, since only one instruction is allowed per line. The indent is just there for readability purposes and is not necessary. 
    
    Having that in mind we therefore split Eq. \ref{Eq.Addexample} the following way:
    \begin{lstlisting}[language=]
.global __start

__start:
    add d, a, b # Find the sum between a and b and put the result in variable d
    add d, d, c # Take the previous result which now lies in d = a + b and add c to it. 
                # Then save the result to the same varible 
    \end{lstlisting}
    We see that we had to use two instructions in order to solve Eq. \ref{Eq.Addexample}. Everything right of the Hashtags are the way we add comments to code, which the computer ignores.
    
    In the previous examples, we made use of a single instruction, the \texttt{add} instruction. There are however 49 base instructions in the RISC-V instruction set, a subset of these is shown in Table \ref{table:RISCVInstructions}. The rest can be referred to in the RISC-V reference card in the appendix.   

    \begin{table}[h!]
        \centering
        \small
        \begin{tabular}{|l|l|l|}
        	\hline
        	Name     & Assembly         & Description (in C) \\ \hline
        	Add      & add rd, rs1, rs2 & rd = rs1 $+$ rs2   \\
        	Subtract & sub rd, rs1, rs2 & rd = rs1 $-$ rs2   \\
        	AND      & and rd, rs1, rs2 & rd = rs1 \& rs2    \\
        	OR       & or rd, rs1, rs2  & rd = rs1 $|$ rs2   \\
        	XOR      & xor rd, rs1, rs2 & rd = rs1 \^{} rs2  \\ \hline
        \end{tabular}
        \caption{This table is a subset of the RISC-V assembly language arithmetic instructions. For the complete base instruction set, please refer to the RISC-V reference card in the appendix.}
        \label{table:RISCVInstructions}
    \end{table}
\newpage
\section{Register}\label{section:Register}
    In a high-level programming language we do not think much, when we declare a variable. For the most part the memory management is hidden for the end user and is taken care of by the compiler. However we do not have this convenience in assembly, where the operands of instructions are much more restricted. 
    
    In the RISC-V architecture we have a special location reserved for the operands build directly into the hardware implementation, which is known as the \textit{register}. It comprises of 32 storage locations for operands, each of which is 64 bits long (this depends on the architecture and can vary in size). These locations are accessed by the instructions, using the numbers 0 through 31 with an x in front e.g. using our addition instruction we type:
    \begin{lstlisting}[language=]
add x18, x12, x13
    \end{lstlisting}
     which finds the sum between values stored in register 12 and 13 and stores the result in register 18.
     
     We do in principle have full control over what we store in these registers, but we will follow the conventions specified in the RISC-V spec sheet, which can be found here\footnote{\url{https://riscv.org/specifications/}} and is summarized in Table \ref{table:RISCVRegister}. Notice that each register also has an ABI (application binary interface) calling convention, so we could also type
     \begin{lstlisting}[language=]
add s2, a2, a3
     \end{lstlisting}
     to access the exact same registers as in the previous example.
     
     \begin{table}[h!]
         \centering
         \begin{tabular}{|l|l|l|}
         	\hline
         	Register & ABI Name & Description                       \\ \hline
         	x0       & zero     & Hard-wired zero                   \\
         	x1       & ra       & Return address                    \\
         	x2       & sp       & Stack pointer                     \\
         	x3       & gp       & Global pointer                    \\
         	x4       & tp       & Thread pointer                    \\
         	x5       & t0       & Temporary/alternate link register \\
         	x6-7     & t1-2     & Temporaries                       \\
         	x8       & s0/fp    & Saved register/frame pointer      \\
         	x9       & s1       & Saved register                    \\
         	x10-11   & a0-1     & Function arguments/return values  \\
         	x12-17   & a2-7     & Function arguments                \\
         	x18-27   & s2-11    & Saved registers                   \\
         	x28-31   & t3-6     & Temporaries                       \\ \hline
         \end{tabular}
         \caption{This table shows the register naming conventions RISC-V architecture. Both the register number and ABI name can be used to access a register.}
         \label{table:RISCVRegister}
     \end{table} 
        
    
\section{Data transfer instructions}\label{section:datatranserinstructions}
    So far our operands in Section \ref{section:RISC-V Asembly} and \ref{section:Register} have been restricted to the register. This approach will however quickly run into its limitations, as only a small amount of data can be stored here. To support larger data sizes and complex data structures, we therefore need a larger storage location for our data, which we call the \textit{memory}.
    
    To access the memory we make use of the \textit{data transfer instructions}. A subset of these are outlined in Table \ref{table:RISCVDataTranferInstructions}.
    
    \begin{table}[h!]
        \centering
        \begin{tabular}{|l|l|l|}
            \hline
            Name             & Assembly         & Description (in C)             \\ \hline
            Load Byte        & lb rd, rs1, imm  & rd = M[rs1$+$imm][0:7]         \\
            Load Half        & lh rd, rs1, imm  & rd = M[rs1$+$imm][0:15]        \\
            Load Word        & lw rd, rs1, imm  & rd = M[rs1$+$imm][0:31]        \\
            Load Doubleword  & ld rd, rs1, imm  & rd = M[rs1$+$imm][0:63]        \\ \hline
            Store Byte       & sb rs1, rs2, imm & M[rs1$+$imm][0:7] = rs2[0:7]   \\
            Store Half       & sh rs1, rs2, imm & M[rs1$+$imm][0:15] = rs2[0:15] \\
            Store Word       & sw rs1, rs2, imm & M[rs1$+$imm][0:31] = rs2[0:31] \\
            Store Doubleword & sd rs1, rs2, imm & M[rs1$+$imm][0:63] = rs2[0:63] \\ \hline
        \end{tabular}
        \caption{This table is a subset of the RISC-V assembly language data transfer instructions. For the complete base instruction set, please refer to the RISC-V reference card behind the appendix.}
        \label{table:RISCVDataTranferInstructions}
    \end{table}

    The memory can be thought of as a big array, where each element is 8 bits or a byte long. In RISC-V the memory is little-endian addressed. This means that if we had to store the doubleword value $8444628$ in memory, which is equal to $10000000 \; 11011010 \; 11010100$ in binary, it would get stored such that rightmost 8 bits would be the \textit{base address} and the subsequent 8 bit chunks offset with respect to the base address. This is illustrated in Figure \ref{fig:memory}.
    
    \begin{figure}[h!]
        \centering
        \subimport{tikz_stuff/}{memory}
        \caption{This figure illustrates the memory, with the data contents inside the blue squares. They each contain 1 byte or 8 bits of data. To the left we see the address of each block, which is offset by 8 due to the number of bits they contain.}
        \label{fig:memory}
    \end{figure}

    Assuming that we in a high-level programming language have associated an array of doublewords, meaning each element in the array is 64 bit, with the variable \texttt{A}. We now want to calculate
    \begin{equation}
        A[2] = x + A[4]
    \end{equation}
    
    where x is stored in register x12 and suppose that the base address of \texttt{A} is saved in register x9. Using the load doubleword instruction, we copy the fourth element into the register 6
    \begin{lstlisting}[language=]
ld x6, x9, 256 # The temporary register x6 gets fourth element in array A.
    \end{lstlisting}
    here the first argument tells the CPU where to save the loaded data, in this case register 6. The second argument tells the instruction, where in memory the array is located. Finally the last argument specifies the offset, so if each doubleword is 64 bits long (each element in A is 64 bits) and we want to access element four in A, we need an offset of 256 ($64\cdot 4=256$).
    
    We then want to do the sum and save it back to the second element in A:
    \begin{lstlisting}[language=]
add x6, x6, x12 # x + A[4]
sd x9, x6, 128  # Store sum in the second element of A 
    \end{lstlisting}
    
    After finding the sum we save the value back into the second element of A using the \texttt{sd} instruction. Here the first argument takes the base address and third argument the offset. The second argument then contains the data to be stored. 
    
    
    
\section{Immediate instructions}
    Up until now we have had to load constants from memory (assuming the values got loaded into memory on startup) to the register everytime we wanted to do an arithmetic operation and what do we do if the memory do not contain the required data? 
    
    As operations with constants are the most frequently used in most programs, dedicated instructions have been added for this purpose. With these the constant is typed into the immediate field \texttt{imm} and is added directly into the instruction itself. 
    
    Using the add immediate instruction \texttt{addi} allows us to add data directly into the register and skip the step of loading constants from memory entirely, like so:
    \begin{lstlisting}[language=]
addi x13, x0, -5 # x13 = 0 - 5
    \end{lstlisting}
    notice that we made use of the zero register, which is hard-wired to zero, to add the number -5 to register 13. A subset of the immediate instructions has been outlined in Table \ref{table:RISCVImmediateInstructions}. It should be noted that any instruction with an \texttt{imm} parameter when refering to the full instruction set behind the appendix, a numerical integer can be inserted just as shown.
    
    \begin{table}[h!]
        \centering
        \begin{tabular}{|l|l|l|}
        	\hline
        	Name          & Assembly          & Description (in C) \\ \hline
        	Add Immediate & addi rd, rs1, imm & rd = rs1 $+$ rs2   \\
        	AND Immediate & andi rd, rs1, imm & rd = rs1 \& rs2    \\
        	OR Immediate  & ori rd, rs1, imm  & rd = rs1 $|$ rs2   \\
        	XOR Immediate & xori rd, rs1, imm & rd = rs1 \^{} rs2  \\ \hline
        \end{tabular}
        \caption{This table is a subset of the RISC-V assembly language arithmetic immediate instructions. For the complete base instruction set, please refer to the RISC-V reference card in the appendix.}
        \label{table:RISCVImmediateInstructions}
    \end{table}
        
\section{Numeral system of a computer}
    In everyday life we are used to using the decimal numeral system, meaning that ten base digits are used to represent a quantity, such as 6 apples. However the decimal numeral system was not chosen to represent numbers in a computer. Here we live in a world were everything consists of signals being high and low voltage or transistors being on or off. Therefore a much more inherent counting system can be chosen to represent numbers in this world, namely the binary numeral system.
    
    The binary digit or bit is the fundamental building block of any computer. A bit requires only two "symbols" to be represented. A high/low signal, on/off transistor, true/false statements or of course, but not limited to, the digits 0 and 1.
    
    In this thesis we label the digits such that the rightmost digit in a number is the zeroth digit and then increasing the index going left. To avoid confusion we will always label which numeral system we are working with e.g. $10_{\;\texttt{two}}$ or $10_{\;\texttt{ten}}$.
    
    Since humans do not usually think in binary, we need a way of converting decimal numbers to binary numbers. We know that any base repressentation of a number can be expressed in base 10 using\footnote{The following derivation is inspired from \\\url{https://math.stackexchange.com/questions/1359770/why-does-the-division-algorithm-work-for-converting-between-number-bases}} (note that the following derivation only works for positive/unsigned numbers) 
    \begin{equation}\label{Eq:baseXToDecimal}
        N_b = \sum_{i=0}^{n} d_i b^i
    \end{equation}
    where $N_b$ is the number to be converted to base 10, $b$ is the target base and $d_i$ is the i'th digit in $N_b$, which satisfies $0 \leq d_k < b$. We can for example expand the number $1101$
    \begin{equation}\label{Eq:bintodecimal}
        1101_{\;\texttt{two}} = (1 \cdot 2^3 + 1 \cdot 2^2 + 0\cdot 2^1 + 1\cdot 2^0)_{\;\texttt{ten}} = 13_{\;\texttt{ten}}
    \end{equation}
    
    Now Eq. \ref{Eq:baseXToDecimal} is invertible by using the remainder theorem, which states that there exists a unique choice of quotient $q$ and remainder $r$ such that 
    \begin{equation}\label{Eq:remainder}
        n = qb+r \quad \text{and} \quad 0 \leq r < b
    \end{equation}
    For example
    \begin{equation}
        13_{\;\texttt{ten}} = (6\cdot 2 + 1)_{\;\texttt{ten}}
    \end{equation}
    Rewriting Eq. \ref{Eq:baseXToDecimal} to
    \begin{equation}\label{Eq:baseXToDecimalRemainderForm}
        N_b = N'_b \cdot b+d_0
    \end{equation}
    where
    \begin{equation}
        N'_b = \sum_{i=0}^{n-1} d_{i+1} b^i
    \end{equation}
    We can see that $d_0$ corresponds to the remainder $r$ and $N'_b$ to the quotient $q$ in \ref{Eq:remainder} .
    Dividing by the base $b$ to get the remainder
    \begin{equation}
        \frac{N_b}{b} = N'_b +\frac{d_0}{b}
    \end{equation}
    which we know corresponds to the leftmost digit in the target repressentation (see Eq. \ref{Eq:bintodecimal}). The same process can be done with the quotient $N'_b$ to gain the remaining digits.
    For example inverting $13_{\;\texttt{ten}}$ back to binary
    \begin{equation}
        \frac{13_{\;\texttt{ten}}}{2_{\;\texttt{ten}}} = 6_{\;\texttt{ten}} + \frac{1_{\;\texttt{ten}} }{2_{\;\texttt{ten}} }
    \end{equation}
    we have the remainder 1, which is the leftmost digit and quotient 6. Reapeating with the quotient
    \begin{equation}
        \frac{6_{\;\texttt{ten}}}{2_{\;\texttt{ten}}} = (3\cdot 2)_{\;\texttt{ten}} + 0_{\;\texttt{ten}} 
    \end{equation}
    we have the remainder 0 and quotient 3. Reapeating again
    \begin{equation}
        \frac{3_{\;\texttt{ten}}}{2_{\;\texttt{ten}}} = 1_{\;\texttt{ten}} + \frac{1_{\;\texttt{ten}} }{2_{\;\texttt{ten}} }
    \end{equation}
    we have the remainder 1 and quotient 1. Reapeating again
    \begin{equation}
    \frac{1_{\;\texttt{ten}}}{2_{\;\texttt{ten}}} = 0_{\;\texttt{ten}} + \frac{1_{\;\texttt{ten}} }{2_{\;\texttt{ten}} }
    \end{equation}
    we have the remainder 1 and quotient 0 and we are done.
    Arranging the remainders in the right order we have
    \begin{equation}
    13_{\;\texttt{ten}} = 1101_{\;\texttt{two}} 
    \end{equation}
    
    We have matched unsigned decimal numbers 0 through 15 with the corresponding unsigned binary value in Table \ref{table:DecimalToBinaryUnsigned}.
    
    \begin{table}[h!]
        \centering
        \begin{tabular}{|l|l||l|l|}
        	\hline
        	Decimal & Binary & Decimal & Binary \\ \hline
        	0       & 0      & 8       & 1000   \\
        	1       & 1      & 9       & 1001   \\
        	2       & 10     & 10      & 1010   \\
        	3       & 11     & 11      & 1011   \\
        	4       & 100    & 12      & 1100   \\
        	5       & 101    & 13      & 1101   \\
        	6       & 110    & 14      & 1110   \\
        	7       & 111    & 15      & 1111   \\ \hline
        \end{tabular}
        \caption{Table of unsigned decimal numbers and their unsigned binary representation.}
        \label{table:DecimalToBinaryUnsigned}
    \end{table} 
    
    \subsection{Signed and Unsigned numbers}
        No computer would be any good for scientific purposes if we are not able to work with negative numbers. Therefore we need a way of representing them in binary. The standard implementation is the \textit{two's complement} representation. Here leading zeros means a positive number and leading ones means a negative number. Using 4 bits we are able to represent numbers -8 to 7 and have been outlined in Table \ref{table:DecimalToBinarySigned}. 
        
        We observe that 
        \begin{equation}\label{Eq:complement}
            x + \bar{x} = -1
        \end{equation}
        where $x$ is a decimal and $\bar{x}$ its complement e.g. $0001_{\; \texttt{two}}+1110_{\; \texttt{two}}=1111_{\; \texttt{two}}$. Rearranging Eq. \ref{Eq:complement}
        \begin{equation}
        \bar{x} + 1 = -x
        \end{equation}
        
        Therefore to negate any number you simply invert all bits and add one e.g. going from -7 to 7
        \begin{equation}\label{Eq:Negationshortcut}
            -7_{\; \texttt{ten}} = 1001_{\; \texttt{two}} \xrightarrow{\text{Invert Bits}} 0110_{\; \texttt{two}} \xrightarrow{\text{+1}} 0111_{\; \texttt{two}} = 7_{\; \texttt{ten}}
        \end{equation}
        Notice however that $-8$ do not have any complement using only 4 bits. In the two's complement representation we will always end up with an extra negative number, if not careful this may cause problems and should be known.
        
        Subtraction of two's complement numbers is very simple. You just add them together and throw away the last bit, which exceeds the size of the bit field. In other words if we got two 4 bit numbers we want to find the sum of and we still have a carry after the 4'th bit, which is supposed to be in the 5'th bit place, we throw away the carry. For example subtracting five with two
        \begin{center}
            \begin{tabular}{c@{\,}c@{\,}c@{\,}c@{\,}c}
            	1 & 1 &   &   &   \\
            	  & \bcancel{0} & \bcancel{1} & 0 & $1_{\; \texttt{two}}$ \\
            	+ & 1 & 1 & 1 & $0_{\; \texttt{two}}$ \\ \hline
            	1 & 0 & 0 & 1 & $1_{\; \texttt{two}}$
            \end{tabular}
        \end{center}
        and throwing away the carry we get $0011_{\; \texttt{two}} = 3_{\; \texttt{ten}}$
        
        
        \begin{table}[h!]
            \centering
            \begin{tabular}{|l|l||l|l|}
            	\hline
            	Decimal & Binary & Decimal & Binary \\ \hline
            	0       & 0000   & -1      & 1111   \\
            	1       & 0001   & -2      & 1110   \\
            	2       & 0010   & -3      & 1101   \\
            	3       & 0011   & -4      & 1100   \\
            	4       & 0100   & -5      & 1011   \\
            	5       & 0101   & -6      & 1010   \\
            	6       & 0110   & -7      & 1001   \\
            	7       & 0111   & -8      & 1000   \\ \hline
            \end{tabular}
            \caption{Table of unsigned decimal numbers and their unsigned binary representation.}
            \label{table:DecimalToBinarySigned}
        \end{table}
    
\section{Instruction representation in binary}\label{section:InstructionRep}
    All base instructions in the RISC-V architecture implemented in this project, \texttt{RV64I}, are 32 bits long, meaning 32 binary digits are used to represent an instruction. This is what all assembly code gets compiled into and is known as \textit{machine code}.
    Going back to our \texttt{add} instruction 
    \begin{lstlisting}[language=, numbers=none]
add x18, x12, x13
    \end{lstlisting}
    in machine code the instruction would look like
    \begin{lstlisting}[language=, numbers=none]
0000000 01101 01100 000    10010 0110011
funct7  x13   x12   funct3 x18   add
    \end{lstlisting}
    where first (remember we count from right) 7 bits tell the CPU what instruction it should perform, in this case \texttt{add}. The next 5 bits tells which register the CPU should save the result to, in this case \texttt{x18}. Bits 12-14 and 25-31 again tell the instruction type in conjunction with the first 7 bits. Bits 15-19 and 20-24 tell which register the first operand and second operand is located respectively, in this case register \texttt{x12} and \texttt{x13}. A subset of instruction formats is shown in Table \ref{table:InstructionFormats}. For a more comprehensive table please refer to the RISC-V instruction set sheet behind the appendix.
    
    \begin{table}[h!]
        \scriptsize
        \begin{tabular} %In the table specs i create 32 even size colums and 1 coloum for text
            {p{0.01mm}p{0.01mm}p{0.01mm}p{0.01mm} p{0.01mm}p{0.01mm}p{0.01mm}p{0.01mm}
                p{0.01mm}p{0.01mm}p{0.01mm}p{0.01mm} p{0.01mm}p{0.01mm}p{0.01mm}p{0.01mm}
                p{0.01mm}p{0.01mm}p{0.01mm}p{0.01mm} p{0.01mm}p{0.01mm}p{0.01mm}p{0.01mm}
                p{0.01mm}p{0.01mm}p{0.01mm}p{0.01mm} p{0.01mm}p{0.01mm}p{0.01mm}p{0.01mm} l}
            %Here we create the row for the bit numbers (only relevant ones are filled for space)
            \multicolumn{1}{c}{31}&&&&&&
            \multicolumn{1}{c}{25}&
            \multicolumn{1}{c}{24}&&&&
            \multicolumn{1}{c}{20}&
            \multicolumn{1}{c}{19}&&&&
            \multicolumn{1}{c}{15}&
            \multicolumn{1}{c}{14}&&
            \multicolumn{1}{c}{12}&
            \multicolumn{1}{c}{11}&&&&
            \multicolumn{1}{c}{7}&
            \multicolumn{1}{c}{6}&&&&&&
            \multicolumn{1}{c}{0}&
            \\
            %Here we create the R-type row
            \cline{0-31} 
            \multicolumn{7}{|c|}{funct7} &
            \multicolumn{5}{c|}{rs2}&
            \multicolumn{5}{c|}{rs1}&
            \multicolumn{3}{c|}{funct3}&
            \multicolumn{5}{c|}{rd}&
            \multicolumn{7}{c|}{opcode}&
            R-type
            \\
            %Here we create the I-type row
            \cline{0-31} 
            \multicolumn{12}{|c|}{imm[11:0]} &
            \multicolumn{5}{c|}{rs1}&
            \multicolumn{3}{c|}{funct3}&
            \multicolumn{5}{c|}{rd}&
            \multicolumn{7}{c|}{opcode}&
            I-type
            \\
            %Here we create the S-type row
            \cline{0-31} 
            \multicolumn{7}{|c|}{imm[11:5]} &
            \multicolumn{5}{c|}{rs2}&
            \multicolumn{5}{c|}{rs1}&
            \multicolumn{3}{c|}{funct3}&
            \multicolumn{5}{c|}{imm[4:0]}&
            \multicolumn{7}{c|}{opcode}&
            S-type
            \\
            \cline{0-31} 
        \end{tabular}
        \caption{A subset of machine code instruction formats, where the \texttt{opcode}, \texttt{funct3} and \texttt{funct7} fields are responsible for instruction identification. The \texttt{rd}, \texttt{rs1}, and \texttt{rs2} fields are operand fields and is responsible for the register destination and register source for the first and second operands respectively. The immediate fields \texttt{imm} are where the constants are stored, where the I-type instruction stores the whole constant in bits 20-31 and S-type splits the constant into bits 7-11 and 25-31.}
        \label{table:InstructionFormats}
    \end{table}

        \subsection{Hexadecimal}
            Now working with 32 bit strings can get quite bothersome. A more compact representation is possible using hexadecimal or base 16. Here a single hexadecimal digit corresponds to 4 binary digits. So we only need 8 hexadecimal digits instead of 32 bits to represent an instruction. All hexadecimal digits and their corresponding binary values are shown in Table \ref{table:Hexadecimal}. 
            
            \begin{table}[h!]
                \centering
                \begin{tabular}{|l|l||l|l||l|l||l|l|}
                	\hline
                	Hexadecimal           & Binary                   & Hexadecimal           & Binary                   & Hexadecimal           & Binary                   & Hexadecimal           & Binary                   \\ \hline
                	$0_{\; \texttt{hex}}$ & $0000_{\; \texttt{two}}$ & $4_{\; \texttt{hex}}$ & $0100_{\; \texttt{two}}$ & $8_{\; \texttt{hex}}$ & $1000_{\; \texttt{two}}$ & $c_{\; \texttt{hex}}$ & $1100_{\; \texttt{two}}$ \\
                	$1_{\; \texttt{hex}}$ & $0001_{\; \texttt{two}}$ & $5_{\; \texttt{hex}}$ & $0101_{\; \texttt{two}}$ & $9_{\; \texttt{hex}}$ & $1001_{\; \texttt{two}}$ & $d_{\; \texttt{hex}}$ & $1101_{\; \texttt{two}}$ \\
                	$2_{\; \texttt{hex}}$ & $0010_{\; \texttt{two}}$ & $6_{\; \texttt{hex}}$ & $0110_{\; \texttt{two}}$ & $a_{\; \texttt{hex}}$ & $1010_{\; \texttt{two}}$ & $e_{\; \texttt{hex}}$ & $1110_{\; \texttt{two}}$ \\
                	$3_{\; \texttt{hex}}$ & $0011_{\; \texttt{two}}$ & $7_{\; \texttt{hex}}$ & $0111_{\; \texttt{two}}$ & $b_{\; \texttt{hex}}$ & $1011_{\; \texttt{two}}$ & $f_{\; \texttt{hex}}$ & $1111_{\; \texttt{two}}$ \\ \hline
                \end{tabular}
                \caption{Table of hexadecimal numbers and their corresponding binary digits.}
                \label{table:Hexadecimal}
            \end{table}
        
            As an example we can convert the add instruction machine code shown in Section \ref{section:InstructionRep} to hexadecimal. Using the lookup table shown in Table \ref{table:Hexadecimal}, we can easily convert the machine code to hex and back, as illustrated in Figure \ref{fig:bintohex}. In many programming languages hexadecimal numbers are usually preceded by a \texttt{0x} such that the hex value shown in Table \ref{fig:bintohex} would look like \texttt{0x00d60d33}.
            
            \begin{figure}[h!]
                \centering
                \subimport{tikz_stuff/}{Bintohex}
                \caption{Using Table \ref{table:Hexadecimal} we can convert the 32 binary string to hexadecimal and vice versa.}
                \label{fig:bintohex}
            \end{figure}
       
\section{Operators}\label{section:Operators}
    In Section \ref{section:InstructionRep} we introduced the machine code representation for instructions. We saw that within the 32-bit strings there existed various fields responsible for instruction identification, target registers and constant fields.
    In this section we take a look at some useful operations, outlined in Table \ref{table:operators}, to address strings of bits or even individual bits within an instruction.
    
    \begin{table}[h!]
        \centering
        \begin{tabular}{|l|c|l|}
        	\hline
        	Decimal                & C operators                & RISC-V Instruction          \\ \hline
        	Logical left shift     & $<<$                      & \texttt{sll}, \texttt{slli} \\ \hline
        	Logical right shift    & $>>$                      & \texttt{srl}, \texttt{srli} \\ \hline
        	Arithmetic right shift & $>>$                      & \texttt{sra}, \texttt{srai} \\ \hline
        	Bitwise AND            & \&                         & \texttt{and}, \texttt{andi} \\ \hline
        	Bitwise OR             & $\mid$                     & \texttt{or}, \texttt{ori}   \\ \hline
        	Bitwise XOR            & \textsuperscript{$\wedge$} & \texttt{xor}, \texttt{xori} \\ \hline
        	Bitwise NOT            & $\sim$                     & \texttt{xori}               \\ \hline
        \end{tabular}
        \caption{Table of logical and bitwise operators in C and corresponding RISC-V Instruction.}
        \label{table:operators}
    \end{table}
    
    \subsection{Shifts}
        To move bits within a string of bits left or right, we make use of the shift operations. There exist two types of shifts, arithmetic and logical.
        The logical shift moves strings of bits in a given direction e.g. we have the 16 bits
        \begin{equation*}
            \text{\Huge 0000\, 0000\, 1101 \, 0110}
        \end{equation*}
        logical shifting left by 9 bits
        \begin{figure*}[h!]
            \centering
            \subimport{tikz_stuff/}{shiftleft}
        \end{figure*}
    \\
        logical shifting right by 9 bits
        \begin{figure*}[h!]
            \centering
            \subimport{tikz_stuff/}{shiftright}
        \end{figure*} 

        When shifting left the bit at the end, bit number 15, gets discarded and at the opposite end, bit number 0, a zero gets inserted. The exact opposite happens when logical shifting right. Notice that the logical shift does not preserve sign, this is what the arithmetic shift is for. Again with the same string of bits
        \begin{equation*}
            \text{\Huge 0000\, 0000\, 1101 \, 0110}
        \end{equation*}
        we perform an arithmetic shift to the left by 9 bits
        \begin{figure*}[h!]
            \centering
            \subimport{tikz_stuff/}{shiftleft}
        \end{figure*}
    \\
        then performing a arithmetic shift to the right by 9 bits
        \begin{figure*}[h!]
            \centering
            \subimport{tikz_stuff/}{shiftrightarithmetic}
        \end{figure*}
        
        When shifting left the bit at the end, bit number 15, gets discarded and at the opposite end, bit number 0, a zero gets inserted. However going right something different happens this time, as the arithmetic shifts preserves sign, so the inserted bit depends on the end bit (bit number 15 here). If the end bit is 1, ones will get inserted and if its 0, zeros will get inserted when shifting right.
        
        We see that the arithmetic shift to the left is exactly the same as the logical left shift, therefore only logical left instruction is needed.
        A beneficial shortcut to know about is shifting bits in the left direction \texttt{i} times multiplies the number by $2^\texttt{i}$ times. Shifting right therefore is the opposite and divides the number by $2^\texttt{i}$.
        
    \subsection{Bitwise logical operations}
        Just like the logical operators \texttt{AND} and \texttt{OR}, which compares two values e.g $7_{\; \texttt{\scriptsize ten}} \; \texttt{AND} \; 4_{\; \texttt{\scriptsize ten}} = \texttt{FALSE}$. We have bitwise versions of the logical operators \texttt{AND} and \texttt{OR}. The bitwise operators work a little different than their logical versions. They compare every bit in a string of bits, so using the operator on the decimals seven and four as before would result in
        
        \begin{center}
            \begin{tabular}{c@{\,}c@{\,}c@{\,}c@{\,}c}
            	   & 0 & 1 & 1 & $1_{\; \texttt{two}}$ \\
            	\& & 0 & 1 & 0 & $0_{\; \texttt{two}}$ \\ \hline
            	   & 0 & 1 & 0 & $0_{\; \texttt{two}}$
            \end{tabular}
        \end{center}
        
        You have to be careful using the bitwise logical operators, as they give unintuitive results when used with decimals as you can see $7_{\; \texttt{\scriptsize ten}} \; \texttt{\&} \; 4_{\; \texttt{\scriptsize ten}} = 4_{\; \texttt{\scriptsize ten}}$.
        
        This operator is very useful when used in conjunction with the logical right shift operator for isolating fields in various strings of bits, for example an instruction. Say we wanted to isolate the register field in the \texttt{add} instruction, 
        \begin{lstlisting}[language=, numbers=none]
                        0000000 01101 01100 000    10010 0110011
                        funct7  rs2   rs1   funct3 rd    add
        \end{lstlisting}
        you simply shift the instruction to the right by 15 bits, such that bits numbers 0-4 now contains register 1 field bits. 
        \begin{lstlisting}[language=, numbers=none]
                        000000000000000000000001101 01100 # shifted right by 15 bits
        \end{lstlisting} 
        Hereafter you perform a bitwise \texttt{AND} instruction with string of bits that contains ones only at position 0-4 
        \begin{lstlisting}[language=, numbers=none]
                        000000000000000000000001101 01100 
                      & 000000000000000000000000000 11111
                    _____________________________________
                        000000000000000000000000000 01100
        \end{lstlisting} 
        everything but the first 5 bits gets canceled out and we end up with the contents of the rs1 field.
        Our 32-bits now contain whole slew of zeros left of the fourth bit. To place string of bits back into the 32-bits we can use the bitwise \texttt{OR} operator for example
        \begin{lstlisting}[language=, numbers=none]
                        000000000000000000000000000 01100
                      | 000000001010110000000000000 00000
                    _____________________________________
                        000000001010110000000000000 01100
        \end{lstlisting}
        Lastly we have the bitwise \texttt{XOR} and \texttt{NOT} operators. Since \texttt{XOR} can be used to perform a \texttt{NOT} operation, only the \texttt{XOR} operator has a dedicated instruction in the RISC-V architecture.
        To emulate the bitwise \texttt{NOT} operation, which inverts every bit, using the \texttt{XOR} operator, you simply compare is to -1, for example using our result from the \texttt{OR} operation
        \begin{lstlisting}[language=, numbers=none]
                        000000001010110000000000000 01100
                      ^ 111111111111111111111111111 11111
                    _____________________________________
                        111111110101001111111111111 10011
        \end{lstlisting}
        
        
    \section{Branching Instructions}
        To produce advanced programs that do more than arithmetic calculations, \textit{branching} instructions are supported by the basic RISC-V instruction set.
        Branching instructions allows the processor to execute instructions not located immediately after the current one based on some condition. 
        
        For example to move 5 instructions back if values stored in registers 17 and 18 are equal the \texttt{beq} instruction would be used. In assembly this would be written like so
        \begin{lstlisting}[language=C]
...
beq x17, x18, -5
        \end{lstlisting}
        
        In higher level programming languages the abstractions for the branching instructions are commonly represented by \texttt{if/else} statements or \texttt{for/while} loops. In an \texttt{if/else} statement the branching instruction would be used to go to a specific place in the program if some condition is met. In a \texttt{for/while} loop branching instructions would be used to jump back and repeat the same part of a program until the condition is met. A subset of the branching instructions are outlined in Table \ref{table:RISCVBranchingInstructions}.
       
        \begin{table}[h!]
            \centering
            \begin{tabular}{|l|l|l|}
            	\hline
            	Name                         & Assembly           & Description (in C)          \\ \hline
            	Branch If Equal              & beq rs1, rs2, imm  & if(rs1 == rs2) PC += imm    \\
            	Branch Not Equal             & bne rs1, rs2, imm  & if(rs1 != rs2) PC += imm    \\
            	Branch Less Than             & blt rs1, rs2, imm  & if(rs1 $<$ rs2) PC += imm   \\
            	Branch Greater Than Or Equal & bgeu rs1, rs2, imm & if(rs1 $\ge$ rs2) PC += imm \\ \hline
            \end{tabular}
            \caption{This table is a subset of the RISC-V assembly language branching instructions. For the complete base instruction set, please refer to the RISC-V reference card in the appendix.}
            \label{table:RISCVBranchingInstructions}
        \end{table}
        
    \section{Jumping Instructions}
        When creating larger programs often same the procedure is used multiple times. An example of this could be calculating $\sin(x)$ multiple places in the code. Instead of rewriting the algorithm for calculating $\sin(x)$ every time it is used, a function can be written once, which we call instead every time it is needed.
        
        In assembly jumping instructions are used for this purpose. Two jumping instructions are supported in the basic instruction set, \texttt{jump and link} and \texttt{jump and link register}. They have been outlined in Table \ref{table:RISCVJumpingInstructions}.
        
        Assuming that the $\sin(x)$ is implemented at some address, jumping to the procedure would be done with a \texttt{jump and link} instruction like so
        \begin{lstlisting}[language=C]
...
jal x1, sinxprocedureaddress
        \end{lstlisting}
        here the address for the next instruction after the current is saved in register \texttt{x1}. To return from the $\sin(x)$ procedure the \texttt{jump and link register} instruction is used as so
        \begin{lstlisting}[language=C]
...
jalr x0, x0, x1
        \end{lstlisting}
        taking advantage of the hard-wired zero in register 0.
        
        \begin{table}[h!]
            \centering
            \begin{tabular}{|l|l|l|}
            	\hline
            	Name                   & Assembly          & Description (in C)          \\ \hline
            	Jump and link          & jal rd, imm       & rd = PC + 4; PC += imm      \\
            	Jump and link register & jalr rd, rs1, imm & rd = PC + 4; PC = rs1 + imm \\ \hline
            \end{tabular}
            \caption{This table shows the RISC-V assembly language jumping instructions. For the complete base instruction set, please refer to the RISC-V reference card in the appendix.}
            \label{table:RISCVJumpingInstructions}
        \end{table}
    
    \section{Compiling simple C code to assembly}
        To close this chapter off an example of a simple \texttt{C} programs compilation into assembly will be shown.
        
        For this example a loop that increments the value, stored in variable \texttt{a} is constructed
        \begin{lstlisting}[language=C]
int a = 0;
for (int i=0; i<10; i++) {
    a+=i;
}
        \end{lstlisting}
        Using any RISC-V compiler the \texttt{C} program can get compiled into its assembly equivalent. It will however look quite different from the code about to be shown depending on the compiler, but the train of thought should still be the same.
        
        In the following the first line adds variable \texttt{a} to register 6 with value 0, the second line add the index \texttt{i} to register 7 with value 0 and constant \texttt{c=10} is added to register 28. 
        
        Line 4 finds the sum between \texttt{a} and \texttt{i}, which gets stored back into register 6 and line 5 increment \texttt{i} by 1.
        
        Lastly to emulate the \texttt{for} loop a branch if less than, \texttt{blt}, instruction has been added to go back to line 4 as long as \texttt{i<10}.
        \begin{lstlisting}[language=C]
ori x6, x0, 0   # a=0
ori x7, x0, 0   # i=0
ori x28, x0, 10 # c=10 (the value i has to be less than in the for loop)
add x6, x6, x7  # a=a+i
addi x7, x7, 1  # i=i+1
blt x7, x28, -2 # if(i < 10) go two instructions back
        \end{lstlisting}
        
        
